\documentclass[11pt, a4paper]{article}  % Set default font and page size


% Load packages and configure them via options
\usepackage[utf8]{inputenc}
\usepackage[left=22mm, right=22mm, top=20mm, bottom=20mm, headheight=40pt, headsep=10pt]{geometry}  % Set margin of page
\usepackage[parfill]{parskip}  % Set spacing between paragraphs
\usepackage{amsmath}  % For the equation environment
\usepackage{graphicx}  % Required for inserting images
\usepackage[dvipsnames]{xcolor}  % Access color codes
\usepackage[colorlinks=true]{hyperref}  % Enable hyperlink
\usepackage{cleveref}  % Enable easy references to sections, figures, and tables
\usepackage{minted}  % Enable code formatting in LaTeX
\usepackage{changepage}  % Required for \adjustwidth
\usepackage{caption}
\usepackage{enumitem}  % Requried for changing configurations for list
\usepackage{lmodern}  % For \texttt font style
\usepackage{booktabs}  % For table formatting
\usepackage{multirow}  % Required for \multirow in tables
\usepackage{emoji}  % Required for Hugging Face emoji
\usepackage[numbers, sort&compress]{natbib}  % For citations and references
\bibliographystyle{unsrtnat}
\usepackage[utf8]{inputenc}

% Define macros, including color codes and macros
\definecolor{YonseiBlue}{HTML}{183772}

% TODO: Replace "CAS2105 Homework 6: Mini AI Pipeline Project \emoji{hugging-face}" with your report title
\title{Mini AI Pipeline Project: Differentiating Clickbaits}  % Title


% TODO: Replace "Your Name (Your Student ID)" with your name and student ID
\author{Yejun Noh (2021147585)}


% Redefine title style
\makeatletter
  \def\@maketitle{%
  \newpage
  \null
  \vskip 2em%
  \begin{flushleft}%
  {\color{YonseiBlue}\rule{\textwidth}{2pt}}
  \vskip 5pt%
  \let \footnote \thanks
    {\Large \bf \@title \par}%
    \vskip 1.5em%
    {\normalsize \bf \lineskip .5em% 
        \@author
        \vskip 2pt%
        \par}
    {\color{YonseiBlue}\rule{\textwidth}{2pt}}
    \vskip 1.5em
  \end{flushleft}%
  }
\makeatother



\begin{document}
\maketitle


\section{Introduction}
\label{sec:introduction}

% --- TODO: Students fill this in ---
% Describe your chosen task and why it is interesting.
% Keep the explanation clear enough for a classmate to understand.

Clickbait headlines are designed to attract attention and encourage users to click on a link, often using exaggerated language.It is therefore necessary to differentiate what is clickbait or not.

This project implements a simple \textbf{AI pipeline} for clickbait detection.

%%%%%%%%%%%%%%%%%%%%%%%%%%%%%%%%%%%%%%%%%%%%%%%%%%%%%%%%%%%%
\section{Task Definition}
\label{sec:definition}
\begin{itemize}
    \item \textbf{Task description:} What are you trying to do? (e.g., ``Classify news headlines into \{sports, politics, tech\}'').
    \item \textbf{Motivation:} Why is this task interesting or useful?
    \item \textbf{Input / Output:} What does the model see, and what should it produce?
    \item \textbf{Success criteria:} How do you know if your system is ``good''?
\end{itemize}

\section{Methods}
\label{sec:methods}

This section includes both the na\"ive baseline and the improved AI pipeline.

\subsection{Na\"ive Baseline}
\label{subsec:baseline}

Implement a simple method that does not rely on heavy models. Examples include:
\begin{itemize}
    \item Keyword-based text classification,
    \item Simple color/shape heuristics for image tasks,
    \item String-overlap–based retrieval.
\end{itemize}

In your report, explain:
\begin{itemize}
    \item How the baseline works,
    \item Why it is considered na\"ive,
    \item Expected failure cases.
\end{itemize}

\subsubsection*{Your Baseline (TODO)}
\begin{itemize}
    \item \textbf{Method description:} TODO
    \item \textbf{Why na\"ive:} TODO
    \item \textbf{Likely failure modes:} TODO
\end{itemize}


\subsection{AI Pipeline}
\label{subsec:pipeline}

Design a small pipeline using one or more pre-trained models. Examples include:

\begin{itemize}
    \item \textbf{Text:} embedding encoder + classifier, or a small transformer model,
    \item \textbf{Retrieval:} embedding model + nearest-neighbor search,
    \item \textbf{Vision:} pre-trained classifier (e.g., ViT-tiny).
\end{itemize}

A typical pipeline contains:
\begin{enumerate}
    \item Preprocessing,
    \item Embedding or representation,
    \item Decision/ranking component,
    \item Optional post-processing.
\end{enumerate}

Fine-tuning large models is not required; inference-only usage is sufficient.

\subsubsection*{Your Pipeline (TODO)}
\begin{itemize}
    \item \textbf{Models used:} TODO
    \item \textbf{Pipeline stages:} TODO
    \item \textbf{Design choices and justification:} TODO
\end{itemize}


%%%%%%%%%%%%%%%%%%%%%%%%%%%%%%%%%%%%%%%%%%%%%%%%%%%%%%%%%%%%
\section{Experiments}
\label{sec:experiments}


\subsection{Datasets}
\label{sec:experiments:datasets}

% --- TODO: Students fill this in ---
% Describe your dataset clearly.

You may use a small public dataset (e.g., from \texttt{datasets}) or construct your own. 
In this section, describe:

\begin{itemize}
    \item \textbf{Dataset source}: where it comes from.
    \item \textbf{Size}: number of examples used.
    \item \textbf{Splits}: how you divided train/validation/test.
    \item \textbf{Preprocessing}: e.g., tokenization, resizing, truncation, normalization.
\end{itemize}

\subsection*{Your Dataset Description (TODO)}
\begin{itemize}
    \item \textbf{Source:} TODO
    \item \textbf{Total examples:} TODO
    \item \textbf{Train/Test split:} TODO
    \item \textbf{Preprocessing steps:} TODO
\end{itemize}


\subsection{Metrics}
\label{sec:experiments:metrics}

Use at least one quantitative metric appropriate for your task:
\begin{itemize}
    \item \textbf{Classification:} accuracy, precision, recall, F1,
    \item \textbf{Retrieval:} precision@k, recall@k,
    \item \textbf{Simple generation:} exact match, ROUGE-1.
\end{itemize}

\textcolor{gray}{It’s worth considering how the metrics you select align with your tasks.}

\subsection{Results}
\label{sec:experiments:results}

Report:
\begin{itemize}
    \item Metric values for baseline vs.\ pipeline,
    \item A results table,
    \item At least three qualitative examples.
\end{itemize}


\begin{center}
\begin{tabular}{lcc}
\toprule
\textbf{Method} & \textbf{Metric 1} & \textbf{Metric 2 (optional)} \\
\midrule
Baseline & TODO & TODO \\
AI Pipeline & TODO & TODO \\
\bottomrule
\end{tabular}
\end{center}

If you want to visualize results with any other than tables, refer to below links
\begin{itemize}
    \item \href{https://matplotlib.org/stable/tutorials/pyplot.html}{Matplotlib official tutorial: Introduction to \texttt{pyplot}}
    \item \href{https://matplotlib.org/stable/gallery/index.html}{Matplotlib example gallery (many bar/line/scatter plots with source code)}
    \item \href{https://www.kaggle.com/code/prashant111/matplotlib-tutorial-for-beginners}{Kaggle notebook: Matplotlib tutorial for beginners (interactive code)}
\end{itemize}
%%%%%%%%%%%%%%%%%%%%%%%%%%%%%%%%%%%%%%%%%%%%%%%%%%%%%%%%%%%%
\section{Reflection and Limitations}
\label{sec:reflection}

Write approximately 6--10 sentences reflecting on:
\begin{itemize}
    \item What worked better than expected,
    \item What failed or was difficult,
    \item How well your metric captured ``quality'',
    \item What you would try next with more time or compute.
\end{itemize}

\subsection*{Your Reflection (TODO)}
% Students write freely here.
TODO


\bibliography{references}

\end{document}
